\section{Conclusion}
In this work, we have considered unary oritatami systems at $\delta = 1$ or $\alpha = 1$.
%As a result, we found that a unary oritatami system does not have Turing universality at $\delta = 1$ and at $\delta \leq 3$, $\alpha = 1$.
%Because a unary oritatami system at $\delta = 1$ and $\alpha = 2$ is able to make an infinity structure, which is zig-zag.
%
%In the case of $\delta = 1$ and $\alpha \geq 3$ it is not able to yield any infinity structures.
%In addition, a unary oritatami system at $\delta = 1$ and $\alpha = 2,3$ is not able to produce any infinite structures.
As a result, we found that a unary oritatami system does not have Turing universality at $\delta = 1$ and at $\delta \leq 3$, $\alpha = 1$.
This non-Turing universality was obtained by the following results.
One is that unary oritatami systems are not able to make any infinite structures at $\delta = 2,3$, $\alpha = 1$ and at $\delta = 1$, $\alpha = 1,3,4$ (Theorems \ref{ttt_inf_d1a4}, \ref{ttt_inf_d1a3}, \ref{thm:d23a1} and due to \cite{DHOPRSST2018}).
The other is that a unary oritatami system can only produce a single type of simple infinite structures, which is zigzag at $\delta = 1$, $\alpha = 2$ (Theorem \ref{ttt_inf_d1a2}).

The case of $\delta \geq 4$ and $\alpha = 1$ remains an open problem.
Our results should be extended to non-unary oritatami systems, that is, characterization Turing universality of oritatami systems with respect to delay, arity or some other parameters such as the number of bead types.
%A non-unary oritatami system is not Turing universal at $\delta = 1$ and $\alpha = 1$ \cite{DHOPRSST2018}.
%Moreover, a non-unary oritatami system is Turing universal at $\delta = 3$ and $\alpha = 3$.